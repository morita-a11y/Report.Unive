\documentclass{jlreq}
\usepackage{url}
\usepackage{listings}
\usepackage{graphicx} % Required for inserting images

\title{第5回 レポート}
\author{森田 蓮}
\date{2025年 1月24日}

\begin{document}

\maketitle

\section{目的}
本実験に用いるfMRIは数ある非侵襲的脳機能イメージング法の中でも信頼性が高く、
基礎・臨床医学のみならず、心理学や工学などさまざまな領域において活用
されている。
本実験の目的は、「fMRIデータの解析でよく用いられているSPM
(Statistical Parametric Mapping)を用いて、fMRIデータの基礎的な解析
手順を実践的に学ぶ」ことである。

\section{脳画像処理1(MRI画像、課題関連脳活動と安静時脳活動)}
\subsection{目的}
解析環境(SPM)の設定、操作法を学び以下の実験のための準備を行う。
また、機能的磁気共鳴画像法(fMRI)の原理を学び説明できるようにする。

\subsection{実験手順}
\begin{enumerate}
    \item MRI, fMRIの原理について理解する <- (2)
    \item サンプルデータの構造(撮像枚数など)を確認する
    \item SPMを用いて、サンプルデータの脳解剖画像を確認する(T1フォルダにある)
    \item SPMを用いて、サンプルデータの脳機能画像を確認する(EPIフォルダにある)
\end{enumerate}
\subsection{結果}

>>>>>>> c99a80c (第5回実験レポート 結果,考察,参考文献以外完了)
\section{脳画像処理2(前処理)}
\subsection{目的}
fMRIデータ解析の前段階となる前処理の基本的な手法を学び、オープンデータセットを用いて実際に前処理を
行う。

\subsection{実験手順}
\begin{enumerate}
    \item Realignment
    \begin{enumerate}
        \item Realign(Estimate)から、Realign & Unwarpを選択
        \item Batch Editorが立ち上がる
        \item Data<-XをダブルクリックもしくはSpecifyしData
の内容を以下のように登録する 
        \item Images<-Xは各被験者のEPIフォルダにあるすべて
の.niiファイル(NifTiファイル)を指定する(合計
204ファイル)。
        \item Runボタン(▷)を押し実行する 
    \end{enumerate}

    \item Coregistration
    \begin{enumerate}
        \item Coregister (Estimate)を選択する
        \item Batch Editorの設定は以下
        \item Reference Image <-XはEPIフォルダのmeansub1\_001.nii,1を指定(被験者1の場合)
        \item Source imageはT1フォルダにあるsub1\_T1.nii,1を選択する
        \item 実行する
    \end{enumerate}
    \item Segmentation
    \begin{enumerate}
        \item Segmentを選択し、以下のBatch Editor設定をする
        \item VolumesはT1フォルダ内のsub1\_T1.nii
        \item Save Bias CorrectedはSave Bias Correctedに変更
        \item Affine RegularisationはEast Asian brainsにする
        \item 次のステップのNormalisationのためにDeformation FieldsをForwardに変更設定する
        \item 実行する
    \end{enumerate}
    \item Normalisation
    \begin{enumerate}
        \item Normalise (Estimate)からNormalise (Write)を選択する
        \item Data<-Xから、Deformation FieldはT1フォルダのy\_sub1\_T1.niiを選択する
        \item Image to Write<-XはEPIフォルダの先頭にuが付いた機能画像(Realign&Unwarp処理をした画
        像)を選択する。
        \item Filterに\^uと入力し、usubから始まる204ファイルを選択する
        \end{enumerate}
    \item Smoothing
    \begin{enumerate}
        \item Smoothを選択する
        \item Images to smooth<-Xにはnormalisation処理後の機能画像(先頭にwがついている)を指定する  \end{enumerate}
\end{enumerate}
>>>>>>> c99a80c (第5回実験レポート 結果,考察,参考文献以外完了)
\subsection{結果}

\section{脳画像処理3(個人解析)}
\subsection{目的}
fMRIデータの個人レベルでの解析手法を学び、オープンデータセットも用いて個人データの解析を
行う。

\subsection{実験手順}
\begin{enumerate}
    \item fMRI model specification
    \begin{enumerate}
        \item Specify 1st-levelを選択する。
        \item Directory<-XでWorkフォルダを選択する
        \item Timing Parametersの設定をする。(Units for design<-XはScans, Interscan interval<-Xは2.5)
        \item Data & Design<-X(ダブルクリック)は、Subject/Sessionの中のScans<-XはEPIフォルダの前
処理済みファイル(swu….nii, 204ファイル)、New: Conditionsを4回クリックする(1PP\_right
条件、1PP\_left条件、3PP\_right条件、3PP\_left条件のため)
        \item 1PP\_right条件については、Name<-Xは1PP\_right、Onsets<-Xはスキャン数を入れる(9)。
    \end{enumerate}

    \item fMRI model estimation
    \begin{enumerate}
        \item Estimateを選択する。
        \item WorkフォルダのSPM.matを選択する
    \end{enumerate}

    \item コントラストの作成と結果の表現
    \begin{enumerate}
        \item Results を選択し、コントラストを定義。
        \item 例: 「3PP条件が1PP条件よりも有意に高い活動を示す」コントラスト設定: -1 -1 1 1 Name: 3PP vs. 1PP
    \end{enumerate}
\end{enumerate}

\subsection{結果}

\section{脳画像処理4(集団解析)}
\subsection{目的}
fMRIデータの集団レベルでの解析手順を学び、オープンデータセットを用いて集団データの解析を行う。

\subsection{実験手順}
\begin{enumerate}
    \item デザインマトリックスの作成
        \begin{enumerate}
            \item SPMを起動し、Batch Editorで以下を設定:Specify 2nd-level Directory に集団解析結果の保存先フォルダを指定
            \item One-sample t-test を選択し、スキャンデータ(例: con\_0001.nii)を5人分選択
        \end{enumerate}
    \item 推定(Estimation)
        \begin{enumerate}
            \item Estimateを選択する
            \item Select SPM.matでParametric Group AnalysisフォルダのSPM.matを選択し、推定を行う
        \end{enumerate}
    \item 解析結果の表示
        \begin{enumerate}
            \item Resultsを選択する。Parametric Group AnalysisにあるSPM.matを選択する
            \item SPM contrast managerが立ち上がるので、個人解析と同様に、contrastを作成する(name: 2nd
                analysis: 3PP vs. 1PPなど、係数は1でよい)
            \item Applying masking: none, threshold, 0.001, & extentthreshold, 10で解析
            \item FWEcの値を読み取り、改めてResultsから解析をし、& extent thresholdをFWEcの値にしてみる
        \end{enumerate}
\end{enumerate}
>>>>>>> c99a80c (第5回実験レポート 結果,考察,参考文献以外完了)
\subsection{結果}

\section{脳画像処理5(その他の解析)}
\subsection{目的}
fMRI(機能的磁気共鳴画像法)を用いた脳の機能的結合性の解析手法を学ぶことである.
安静時脳活動におけるデフォルトモードネットワーク(DMN)の機能的結合性を解析することで、脳内ネットワークの理解を深めることを目指す.
\subsection{実験手順}
\begin{enumerate}
    \item デフォルトモードネットワーク(DMN)の解析
    \begin{enumerate}
        \item DMNに属する主要な脳領域(PCC, MPFC, LPなど)を対象に、機能的結合性を解析
        \item Seed-to-Voxel解析では、あるROIの主成分と全脳のその他のvoxel間の結合を検証
        \item ROI-to-ROI解析では、関心領域間の機能的結合を検討
    \end{enumerate}

    \item PPI解析 
    \begin{enumerate}
        \item 課題条件(例えば、3PPと1PP)に基づき、脳領域間の機能的結合に対する心理的プロセスの影響を解析
    \end{enumerate}

    \item 結果の確認と応用
    \begin{enumerate}
        \item 解析結果(脳領域名、座標、結合性の値)を可視化
        \item 多重比較補正(FDR correction)や仮説検証を行い、解析をさらに発展
    \end{enumerate}
\end{enumerate}
>>>>>>> c99a80c (第5回実験レポート 結果,考察,参考文献以外完了)
\subsection{結果}



\section{考察}



\section*{参考文献}


\end{document}
