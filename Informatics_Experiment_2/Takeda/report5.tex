\documentclass{jlreq}
\usepackage{url}
\usepackage{listings}
\usepackage{graphicx} % Required for inserting images

\title{第5回 レポート}
\author{森田 蓮}
\date{2025年 1月24日}

\begin{document}

\maketitle

\section{目的}
本実験に用いるfMRIは数ある非侵襲的脳機能イメージング法の中でも信頼性が高く、
基礎・臨床医学のみならず、心理学や工学などさまざまな領域において活用
されている。
本実験の目的は、「fMRIデータの解析でよく用いられているSPM
(Statistical Parametric Mapping)を用いて、fMRIデータの基礎的な解析
手順を実践的に学ぶ」ことである。

\section{脳画像処理1(MRI画像、課題関連脳活動と安静時脳活動)}
\subsection{目的}
解析環境(SPM)の設定、操作法を学び以下の実験のための準備を行う。
また、機能的磁気共鳴画像法(fMRI)の原理を学び説明できるようにする。
\subsection{方法}

\subsection{結果}

\section{脳画像処理2(前処理)}
\subsection{目的}
fMRIデータ解析の前段階となる前処理の基本的な手法を学び、オープンデータセットを用いて実際に前処理を
行う。
\subsection{方法}

\subsection{結果}

\section{脳画像処理3(個人解析)}
\subsection{目的}
fMRIデータの個人レベルでの解析手法を学び、オープンデータセットも用いて個人データの解析を
行う。
\subsection{方法}

\subsection{結果}

\section{脳画像処理4(集団解析}
\subsection{目的}
fMRIデータの集団レベルでの解析手順を学び、オープンデータセットを用いて集団データの解析を行う。

\subsection{方法}

\subsection{結果}

\section{脳画像処理5(その他の解析)}
\subsection{目的}

\subsection{方法}

\subsection{結果}



\section{考察}



\section*{参考文献}


\end{document}