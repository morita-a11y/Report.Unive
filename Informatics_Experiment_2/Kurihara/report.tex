\documentclass{jlreq}
\usepackage{url}
\usepackage{graphicx}
\usepackage{listings}
\usepackage{xcolor}
\usepackage{multicol}
 
\lstset{
    basicstyle=\ttfamily\small,
    numbers=left,
    numberstyle=\tiny\color{gray},
    keywordstyle=\color{blue},
    commentstyle=\color{green!50!black},
    stringstyle=\color{orange},
    frame=single,
    tabsize=4,
    breaklines=true
}

\title{第6回 実験レポート}
\author{森田 蓮}
\date{2025年 2月11日}

\begin{document}

\maketitle

\section{目的}
VRやサイバーフィジカル空間を中心としたメディア技術として、
最も身近なWEB関連技術について実験を行い理解を深めることを目的とする.

\section{考え方の説明}
\subsection{HTML/CSS}
HTMLはwebページの構造を作り,CSSはHTMLでコーディングされた要素に色や枠,
文字の大きさなどのデザインをつける物である.

aa
\subsection{JavaScript}

\subsection{2D Canvas}

\subsection{3D Canvas (WebGL)}



\section{行列計算について}



\begin{thebibliography}{9}

\bibitem{video1}
YouTube, ``MRI and fMRI Principles Explained'', \url{https://www.youtube.com/watch?v=NlYXqRG7lus}, accessed January 25, 2025.

\bibitem{video2}
YouTube, ``Understanding MRI and fMRI'', \url{https://www.youtube.com/watch?v=jLnuPKhKXVM}, accessed January 25, 2025.

\bibitem{video3}
YouTube, ``MRI vs fMRI: What's the Difference?'', \url{https://www.youtube.com/watch?v=rJjHjnzmvDI}, accessed January 25, 2025.

\end{thebibliography}

\section*{付録}

\end{document}