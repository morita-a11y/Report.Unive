\documentclass{jlreq}
\usepackage{url}
\usepackage{graphicx}
\usepackage{listings}
\usepackage{xcolor}
\usepackage{multicol}
 
\title{第5回 実験レポート}
\author{森田 蓮}
\date{2025年 1月27日}

\begin{document}

\maketitle

\section{目的}
本レポートでは、TCP を利用したクライアントプログラムの動作を確認し、その仕様や実装について考察する。

\section{方法}

\begin{enumerate}
    \item クライアントプログラムを実装し、TCP を用いた通信を行う。
    \item プログラムの仕様を整理し、コードの詳細を分析する。
    \item 実行結果を記録し、動作の確認を行う。
    \item 実装の課題点や改善点を考察する。
\end{enumerate}

\section{仕様}

本プログラムは、TCP を利用したクライアントプログラムであり、サーバーとの通信を行う。標準入力からの入力をサーバーへ送信し、サーバーからの応答を標準出力に表示する。

\begin{itemize}
    \item Socket API を用い,コマンドライン引数で指定されたホスト名とポート番号のサーバーにTCPで接続する.
    \item ホスト名の名前解決には getaddrinfo を用いる.
    \item 入出力には read(2), write(2) を用いる.
    \item read(2), write(2) は講義中に説明した注意を守ること.そのような write(2) 呼び出しを関数として独立させて,他からは write(2) 直接ではなく,その関数を呼び出すようにするとよい.
    \item サーバーと接続したソケットおよび標準入力の2つを select() で監視することが望ましい.
    \item もし標準入力から入力があったら,その内容をサーバーへ送信する.
    \item もしサーバーからデータを受信したら,それを標準出力へ出力する.
    \item どちらかが EOF になるか,エラーが発生するまで上記を繰り返す.
    \item サーバーとの接続を切断して終了する.
\end{itemize}

\section{プログラムの説明}

\subsection{safe_write 関数}

\begin{verbatim}
    ssize_t safe_write(int fd, const void *buf, size_t count)
\end{verbatim}

この関数は、安全にデータを送信するための関数であり、writeシステムコールが途中で中断された場合にもリトライする。

\subsection{ソケットの作成と接続}

\begin{verbatim}
    sockfd = socket(r->ai_family, r->ai_socktype, r->ai_protocol);
    connect(sockfd, r->ai_addr, r->ai_addrlen);
\end{verbatim}

指定されたサーバーに対してソケットを作成し、接続を試みる。

\subsection{select を使用した非同期通信}

\begin{verbatim}
    FD_ZERO(&read_fds);
    FD_SET(STDIN_FILENO, &read_fds);
    FD_SET(sockfd, &read_fds);
    select(max_fd + 1, &read_fds, NULL, NULL, NULL);
\end{verbatim}

selectを用いることで、標準入力とサーバーの両方のデータを同時に監視し、データが到着した場合に処理を実行する。

\section{考察}
\begin{itemize}
    \item  select を使用することで、非同期処理が可能になり、標準入力とソケットを並行して監視できる。
    \item  safe_write を導入することで、データ送信の安全性を向上させた。
    \item IPv4 および IPv6 の両方に対応しており、汎用性の高いクライアントプログラムになっている。
    \item エラー処理を適切に行うことで、安定した通信が実現できる。
\end{itemize}

\section{感想}

本プログラムの開発を通じて、ソケットプログラミングの基本と、selectを利用した非同期処理の重要性を学ぶことができた。
また、実際のネットワークプログラムではエラーハンドリングが非常に重要であることを再認識した。


\end{document}