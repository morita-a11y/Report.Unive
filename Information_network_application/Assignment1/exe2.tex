\documentclass{jlreq}
\usepackage{url}
\usepackage{graphicx}
\usepackage{listings}
\usepackage{xcolor}
\usepackage{multicol}

\lstset{
    basicstyle=\ttfamily\small,
    numbers=left,
    numberstyle=\tiny\color{gray},
    keywordstyle=\color{blue},
    commentstyle=\color{green!50!black},
    stringstyle=\color{orange},
    frame=single,
    tabsize=4,
    breaklines=true
}

\title{ネットワーク応用 課題2}
\author{森田 蓮}
\date{2025年 1月08日}

\begin{document}

\maketitle

\section{追加した内容の説明}
\begin{verbatim}
    fp -> count = 0;
    fp -> index = 0;
\end{verbatim}
ファイルを開くときの,バッファ内の文字数を初期化する.    

\begin{verbatim}
    r = close(fp->fd);
    free(fp);
\end{verbatim}

close システムコールを使用して,
fp->fd に格納されたファイルディスクリプタを閉じる.
malloc によって動的に割り当てられた fp のメモリを
free 関数で解放する.


\begin{verbatim}
    if (fp->index == fp->count) {
  size = read(fp->fd, fp->buffer, MyBufferSize);
  if (size <= 0) {
    return EOF;
  }
  fp->count = size;
  fp->index = 0;
}
c = fp->buffer[fp->index++];
\end{verbatim}
index が count に等しい場合,バッファー内のデータがすべて読み取られ
バッファーが空であることを判定しread システムコールを使用して,fp->fd
に対応するファイルから
 MyBufferSize バイトのデータを fp->buffer に読み込む.

 if (size <= 0) で読み込みエラーまたは EOF を確認する.
 fp->buffer の index 番目の要素を読み取り,index を
インクリメントして次の文字を指すように更新する.


\section{実行例}
\begin{lstlisting}[language={}, caption=実行結果]
    ren@Renking:~/Github/Report.Unive/Information_network_application/Exe2$ cat test.txt
    abcdefghijk
    ren@Renking:~/Github/Report.Unive/Information_network_application/Exe2$ ./a.out test.txt
    count:0 index:0  | 00 00 00 00 00 |
    count:5 index:1  | 61 62 63 64 65 |
    c:61
    count:5 index:2  | 61 62 63 64 65 |
    c:62
    count:5 index:3  | 61 62 63 64 65 |
    c:63
    count:5 index:4  | 61 62 63 64 65 |
    c:64
    count:5 index:5  | 61 62 63 64 65 |
    c:65
    count:5 index:1  | 66 67 68 69 6a |
    c:66
    count:5 index:2  | 66 67 68 69 6a |
    c:67
    count:5 index:3  | 66 67 68 69 6a |
    c:68
    count:5 index:4  | 66 67 68 69 6a |
    c:69
    count:5 index:5  | 66 67 68 69 6a |
    c:6a
    count:2 index:1  | 6b 0a 68 69 6a |
    c:6b
    count:2 index:2  | 6b 0a 68 69 6a |
    c:0a
    count:2 index:2  | 6b 0a 68 69 6a |
    \end{lstlisting}

\section{考察}
今回のバッファリングの入力機構を作成したが,
バッファーが空になってreadを呼び出す場合や毎回readを呼び出す場合での
差がわからなかった.
数文字程度では実行時間に差が出ないと考える.
\section{感想}
C言語でファイル操作,ポインタの操作を今までしたことがなかったため
,個人的に難易度がとても高かった.
今回の課題を通してファイル操作を学ぶことができて
成長できると確信している.
これからは,書いたコードにデータをそのまま打ち込むのは
ナンセンスであると思い情報学群生活を送りたい.



\end{document}