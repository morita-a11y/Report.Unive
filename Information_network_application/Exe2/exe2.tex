\documentclass{jlreq}
\usepackage{url}
\usepackage{amsmath,amssymb}
\title{情報ネットワーク応用 演習2 - レポート課題}
\author{森田 蓮}
\date{2025年 1月20日}

\begin{document}

\maketitle

\section{演習手順}
\begin{enumerate}
    \item Nmapを利用して,カメラが動作しているサービスを調べる.
    \item ブルートフォース攻撃を行い管理画面の認証を突破する.
    \item カメラを乗っ取り動作させることを確認する.
    \item パスワードを変更してブルートフォース攻撃でパスワードが漏れないことを確認する.
    \item WEBカメラから機器の乗っ取りを行う.
\end{enumerate}

\section{演習2で学んだこと}

\section{感想}
今回の演習では班の人数が少ないこともありとても私たちの班は自分たちの理解がしやすい演習になった.
本演習では,序盤スライドの冒頭部分のコマンドの説明を見て作業しており私たちの欲しい情報が上手く出力されず
時間を浪費してしまった.
やはりコマンドのオプションは理解してから使うべきであると思う.
わからないコマンドを利用することは非常に悪手であり説明を読んで理解するには
ネットワークであるならネットワークの知識を身に着ける必要がある.

\end{document}