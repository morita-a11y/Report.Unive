\documentclass{jlreq}
\usepackage{url}
\usepackage{graphicx}
\usepackage{listings}
\usepackage{xcolor}
\usepackage{multicol}
 
\lstset{
    basicstyle=\ttfamily\small,
    numbers=left,
    numberstyle=\tiny\color{gray},
    keywordstyle=\color{blue},
    commentstyle=\color{green!50!black},
    stringstyle=\color{orange},
    frame=single,
    tabsize=4,
    breaklines=true
}

\title{演習3 - レポート課題}
\author{森田 蓮}
\date{2025年 1月28日}

\begin{document}

\maketitle

\section{目的}
本演習では,バッファオーバーフロー (Buffer Overflow) 脆弱性を悪用した攻撃手法と,その対策について学ぶ.
この脆弱性は,プログラムが入力データのサイズを適切に検証せずにメモリに格納することで発生する.
本レポートでは,脆弱性の原因,攻撃手法,及びその防止策について学んだことを説明する.

\section{学んだことを}
バッファオーバーフローとは,固定サイズのバッファ(配列など)に対して,許容サイズを超えるデータを書き込むことにより,
メモリの隣接領域が上書きされる問題を指す.
gets()関数は入力データの長さを制限しないため,64バイトを超えるデータを入力すると,
スタックに保存されている他のデータ(例えばリターンアドレス)が上書きされる可能性がある.

\section{攻撃手法}
演習では,以下の手順を通じて攻撃を実行した.
\begin{enumerate}
    \item ソースコードの解析: ソースコードを確認し,
    脆弱性のある箇所(gets()関数の使用)を特定できた.
    また,未使用のshell()関数が含まれており,
    これを悪用してシェルを起動することが可能であると判断した.
    \item アドレスの特定:実行ファイルを逆アセンブルして,shell()関数とリターンアドレスの位置を特定する.
    \item 攻撃コードの作成:Pwntoolsを使用して,攻撃コードを作成する.
\end{enumerate}

\section{防御手法}
gets()関数の代わりにfgets()関数を使用し,入力データの長さを制限する.
この操作により,バッファサイズ以上の文字列を読み込まないようにできた.

\section{感想}
今回の演習は,とても難しくこのレポートを書いていて調べている中で学べることが
多い演習だったと感じた.
座学だけではイメージができないことが実際に動作するところを見ることで
危険性が理解できた.
自分で作る任意のコードを実行できることはとても怖いことだとこの3回の
演習を行って感じた.
危険な関数があることを知ってなぜ危険なのかを理解できた.
これからのたくさんのコーディングを行う際に危険な関数を使ってユーザに
不利益を被らないような技術者になりたいと思う.


\end{document}
