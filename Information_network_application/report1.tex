\documentclass{jlreq}
\usepackage{url}
\usepackage{amsmath,amssymb}
\title{情報ネットワーク応用 演習1 - レポート課題}
\author{森田 蓮}
\date{2024年 12月24日}

\begin{document}

\maketitle

\section{OSコマンドインジェクションの危険性と対策}
OSコマンドインジェクションは,外部からの入力をOSコマンドに渡すときに開発者が予期しない
不正な入力をOSが実行してしまうという脆弱性である.
今回の演習の例で言えばサーバーのPythonで書かれたソースコードでユーザーからの入力を
そのままdigコマンドに渡してShellを利用して実行していることがOSコマンドインジェクションを
引き起こす原因となることがわかる.
<<<<<<< HEAD

=======
>>>>>>> 3f93c73057df0268cc2a5f8e1e411c833e65cebe
また,講義の話ではC言語で書くサーバーのコードには関数gets()を利用すること制限なしで文字列を
受け取る.
この関数gets()もOSコマンドインジェクションを引き起こす危険な記述である.

これらの対策としてShellを利用せずにコマンドを実行するように修正を加えることや,
受け取る文字列に制限を加えOSコマンドで特別な意味を持つ文字を使えないようにする方法がある.

\begin{verbatim}
VALID_INPUT_PATTERN = r"^[a-zA-Z0-9.-]+$
\end{verbatim}

VALID_INPUT_PATTERNでユーザーからの入力を制限し条件を正規表現で
表し指定する.

\begin{verbatim}
    is_valid_input(user_input)
\end{verbatim}

この関数で制限された入力に対しTrue or Falseで条件に一致するかを
判定する.
ここではIf文を用いてFalseの場合は入力を弾き拒否する.
このような対策を講じることでOSコマンドインジェクションの対策として
有効である.



\section{感想}
今回の第1回演習では自分が講義を聞いただけでは実際にどんな動きをするのかわからなかったことが
実際に手を動かして見ることで納得できることが多かった.
OSコマンドインジェクションの脆弱性をついた発生理由とその影響が演習を通してどの部分がサーバー
側で脅威になることを理解できた.
サーバーを設計するときにユーザーの入力を信じてはいけないと思う.
安全なコードを書くためには前提になる知識が必須で知らずにやってしまうのが
一番悪だと考える.
今回の演習はトラブルがあり自分が納得するまで攻撃を考えてすることができなかったことが残念
だった.
TAさんがトラブルを直している姿がとてもお手本となる大学院生と思った.
私も専門的なことを身に着けてたいなとモチベーションになりとてもいい機会で
あったと思う.
より勉強をして今回Dokerに対する理解が足りていないのでもっと深めたいと
思う.
脆弱性に対する理解が足りていないと実感するので
様々な脆弱性について考え,自分が開発の立場になる際に
自分はしていなくてもほかの人が作ったコードを見て
善し悪しの判断ができるようにしていきたいと思う.
自分のためになる演習になるように努力を続けたい.


\end{document}