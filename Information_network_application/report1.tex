\documentclass{jlreq}
\usepackage{url}
\usepackage{amsmath,amssymb}
\title{情報ネットワーク応用 演習1 - レポート課題}
\author{森田 蓮}
\date{2024年 12月24日}

\begin{document}

\maketitle

\section{OSコマンドインジェクションの危険性と対策}
OSコマンドインジェクションは,外部からの入力をOSコマンドに渡すときに開発者が予期しない
不正な入力をOSが実行してしまうという脆弱性である.
今回の演習の例で言えばサーバーのPythonで書かれたソースコードでユーザーからの入力を
そのままdigコマンドに渡してShellを利用して実行していることがOSコマンドインジェクションを
引き起こす原因となることがわかる.
また,講義の話ではC言語で書くサーバーのコードには関数gets()を利用すること制限なしで文字列を
受け取る.
この関数gets()もOSコマンドインジェクションを引き起こす危険な記述である.

これらの対策としてShellを利用せずにコマンドを実行するように修正を加えることや,
受け取る文字列に制限を加えOSコマンドで特別な意味を持つ文字を使えないようにする方法がある.




\section{感想}
今回の第1回演習では自分が講義を聞いただけでは実際にどんな動きをするのかわからなかったことが
実際に手を動かして見ることで納得できることが多かった.
今回の演習はトラブルがあり自分が納得するまで攻撃を考えてすることができなかったことが残念
だった.
TAさんがトラブルを直している姿がとてもお手本となる大学院生と思った.
私も専門的なことを身に着けてたいなとモチベーションになりとてもいい機会で
あったと思う.
より勉強をして今回Dokerに対する理解が足りていないのでもっと深めたいと
思う.


\end{document}